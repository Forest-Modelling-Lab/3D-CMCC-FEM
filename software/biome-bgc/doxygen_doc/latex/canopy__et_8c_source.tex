\hypertarget{canopy__et_8c_source}{}\section{canopy\+\_\+et.\+c}
\label{canopy__et_8c_source}\index{c\+:/\+R\+E\+P\+O/biomebgc-\/4.\+2/src/bgclib/canopy\+\_\+et.\+c@{c\+:/\+R\+E\+P\+O/biomebgc-\/4.\+2/src/bgclib/canopy\+\_\+et.\+c}}

\begin{DoxyCode}
00001 \textcolor{comment}{/* }
00002 \textcolor{comment}{canopy\_et.c}
00003 \textcolor{comment}{A single-function treatment of canopy evaporation and transpiration}
00004 \textcolor{comment}{fluxes.  }
00005 \textcolor{comment}{}
00006 \textcolor{comment}{*-*-*-*-*-*-*-*-*-*-*-*-*-*-*-*-*-*-*-*-*-*-*-*-*}
00007 \textcolor{comment}{Biome-BGC version 4.2 (final release)}
00008 \textcolor{comment}{See copyright.txt for Copyright information}
00009 \textcolor{comment}{*-*-*-*-*-*-*-*-*-*-*-*-*-*-*-*-*-*-*-*-*-*-*-*-*}
00010 \textcolor{comment}{*/}
00011 
00012 \textcolor{preprocessor}{#include "bgc.h"}
00013 
\hypertarget{canopy__et_8c_source_l00014}{}\hyperlink{canopy__et_8c_a5882089955eaa9ed13e938c6eb383b11}{00014} \textcolor{keywordtype}{int} \hyperlink{canopy__et_8c_a5882089955eaa9ed13e938c6eb383b11}{canopy\_et}(\textcolor{keyword}{const} metvar\_struct* metv, \textcolor{keyword}{const} epconst\_struct* epc, 
00015 epvar\_struct* epv, wflux\_struct* wf, \textcolor{keywordtype}{int} verbose)
00016 \{
00017     \textcolor{keywordtype}{int} ok=1;
00018     \textcolor{keywordtype}{double} gl\_bl, gl\_c, gl\_s\_sun, gl\_s\_shade;
00019     \textcolor{keywordtype}{double} gl\_e\_wv, gl\_t\_wv\_sun, gl\_t\_wv\_shade, gl\_sh;
00020     \textcolor{keywordtype}{double} gc\_e\_wv, gc\_sh;
00021     \textcolor{keywordtype}{double} tday;
00022     \textcolor{keywordtype}{double} tmin;
00023     \textcolor{keywordtype}{double} dayl;
00024     \textcolor{keywordtype}{double} vpd,vpd\_open,vpd\_close;
00025     \textcolor{keywordtype}{double} psi,psi\_open,psi\_close;
00026     \textcolor{keywordtype}{double} m\_ppfd\_sun, m\_ppfd\_shade;
00027     \textcolor{keywordtype}{double} m\_tmin, m\_psi, m\_co2, m\_vpd, m\_final\_sun, m\_final\_shade;
00028     \textcolor{keywordtype}{double} proj\_lai;
00029     \textcolor{keywordtype}{double} canopy\_w;
00030     \textcolor{keywordtype}{double} gcorr;
00031     
00032     \textcolor{keywordtype}{double} e, cwe, t, trans, trans\_sun, trans\_shade, e\_dayl,t\_dayl; 
00033     pmet\_struct pmet\_in;
00034 
00035     \textcolor{comment}{/* assign variables that are used more than once */}
00036     tday =      metv->tday;
00037     tmin =      metv->tmin;
00038     vpd =       metv->vpd;
00039     dayl =      metv->dayl;
00040     psi =       epv->psi;
00041     proj\_lai =  epv->proj\_lai;
00042     canopy\_w =  wf->prcp\_to\_canopyw;
00043     psi\_open =  epc->psi\_open;
00044     psi\_close = epc->psi\_close;
00045     vpd\_open =  epc->vpd\_open;
00046     vpd\_close = epc->vpd\_close;
00047 
00048     \textcolor{comment}{/* temperature and pressure correction factor for conductances */}
00049     gcorr = pow((metv->tday+273.15)/293.15, 1.75) * 101300/metv->pa;
00050     
00051     \textcolor{comment}{/* calculate leaf- and canopy-level conductances to water vapor and}
00052 \textcolor{comment}{    sensible heat fluxes */}
00053     
00054     \textcolor{comment}{/* leaf boundary-layer conductance */}
00055     gl\_bl = epc->gl\_bl * gcorr;
00056     
00057     \textcolor{comment}{/* leaf cuticular conductance */}
00058     gl\_c = epc->gl\_c * gcorr;
00059     
00060     \textcolor{comment}{/* leaf stomatal conductance: first generate multipliers, then apply them}
00061 \textcolor{comment}{    to maximum stomatal conductance */}
00062     \textcolor{comment}{/* calculate stomatal conductance radiation multiplier: }
00063 \textcolor{comment}{    *** NOTE CHANGE FROM BIOME-BGC CODE ***}
00064 \textcolor{comment}{    The original Biome-BGC formulation follows the arguments in }
00065 \textcolor{comment}{    Rastetter, E.B., A.W. King, B.J. Cosby, G.M. Hornberger, }
00066 \textcolor{comment}{       R.V. O'Neill, and J.E. Hobbie, 1992. Aggregating fine-scale }
00067 \textcolor{comment}{       ecological knowledge to model coarser-scale attributes of }
00068 \textcolor{comment}{       ecosystems. Ecological Applications, 2:55-70.}
00069 \textcolor{comment}{}
00070 \textcolor{comment}{    gmult->max = (gsmax/(k*lai))*log((gsmax+rad)/(gsmax+(rad*exp(-k*lai))))}
00071 \textcolor{comment}{}
00072 \textcolor{comment}{    I'm using a much simplified form, which doesn't change relative shape}
00073 \textcolor{comment}{    as gsmax changes. See Korner, 1995.}
00074 \textcolor{comment}{    */}
00075     \textcolor{comment}{/* photosynthetic photon flux density conductance control */}
00076     m\_ppfd\_sun = metv->ppfd\_per\_plaisun/(PPFD50 + metv->ppfd\_per\_plaisun);
00077     m\_ppfd\_shade = metv->ppfd\_per\_plaishade/(PPFD50 + metv->ppfd\_per\_plaishade);
00078 
00079     \textcolor{comment}{/* soil-leaf water potential multiplier */}
00080     \textcolor{keywordflow}{if} (psi > psi\_open)    \textcolor{comment}{/* no water stress */}
00081         m\_psi = 1.0;        
00082     \textcolor{keywordflow}{else}
00083     \textcolor{keywordflow}{if} (psi <= psi\_close)   \textcolor{comment}{/* full water stress */}
00084     \{
00085         m\_psi = 0.0;      
00086     \}
00087     \textcolor{keywordflow}{else}                   \textcolor{comment}{/* partial water stress */}
00088         m\_psi = (psi\_close - psi) / (psi\_close - psi\_open);
00089 
00090     \textcolor{comment}{/* CO2 multiplier */}
00091     m\_co2 = 1.0;
00092 
00093     \textcolor{comment}{/* freezing night minimum temperature multiplier */}
00094     \textcolor{keywordflow}{if} (tmin > 0.0)        \textcolor{comment}{/* no effect */}
00095         m\_tmin = 1.0;
00096     \textcolor{keywordflow}{else}
00097     \textcolor{keywordflow}{if} (tmin < -8.0)       \textcolor{comment}{/* full tmin effect */}
00098         m\_tmin = 0.0;
00099     \textcolor{keywordflow}{else}                   \textcolor{comment}{/* partial reduction (0.0 to -8.0 C) */}
00100         m\_tmin = 1.0 + (0.125 * tmin);
00101     
00102     \textcolor{comment}{/* vapor pressure deficit multiplier, vpd in Pa */}
00103     \textcolor{keywordflow}{if} (vpd < vpd\_open)    \textcolor{comment}{/* no vpd effect */}
00104         m\_vpd = 1.0;
00105     \textcolor{keywordflow}{else}
00106     \textcolor{keywordflow}{if} (vpd > vpd\_close)   \textcolor{comment}{/* full vpd effect */}
00107         m\_vpd = 0.0;
00108     \textcolor{keywordflow}{else}                   \textcolor{comment}{/* partial vpd effect */}
00109         m\_vpd = (vpd\_close - vpd) / (vpd\_close - vpd\_open);
00110 
00111     \textcolor{comment}{/* apply all multipliers to the maximum stomatal conductance */}
00112     m\_final\_sun = m\_ppfd\_sun * m\_psi * m\_co2 * m\_tmin * m\_vpd;
00113     \textcolor{keywordflow}{if} (m\_final\_sun < 0.00000001) m\_final\_sun = 0.00000001;
00114     m\_final\_shade = m\_ppfd\_shade * m\_psi * m\_co2 * m\_tmin * m\_vpd;
00115     \textcolor{keywordflow}{if} (m\_final\_shade < 0.00000001) m\_final\_shade = 0.00000001;
00116     gl\_s\_sun = epc->gl\_smax * m\_final\_sun * gcorr;
00117     gl\_s\_shade = epc->gl\_smax * m\_final\_shade * gcorr;
00118     
00119     \textcolor{comment}{/* calculate leaf-and canopy-level conductances to water vapor and}
00120 \textcolor{comment}{    sensible heat fluxes, to be used in Penman-Monteith calculations of}
00121 \textcolor{comment}{    canopy evaporation and canopy transpiration. */}
00122     
00123     \textcolor{comment}{/* Leaf conductance to evaporated water vapor, per unit projected LAI */}
00124     gl\_e\_wv = gl\_bl;
00125         
00126     \textcolor{comment}{/* Leaf conductance to transpired water vapor, per unit projected}
00127 \textcolor{comment}{    LAI.  This formula is derived from stomatal and cuticular conductances}
00128 \textcolor{comment}{    in parallel with each other, and both in series with leaf boundary }
00129 \textcolor{comment}{    layer conductance. */}
00130     gl\_t\_wv\_sun = (gl\_bl * (gl\_s\_sun + gl\_c)) / (gl\_bl + gl\_s\_sun + gl\_c);
00131     gl\_t\_wv\_shade = (gl\_bl * (gl\_s\_shade + gl\_c)) / (gl\_bl + gl\_s\_shade + gl\_c);
00132 
00133     \textcolor{comment}{/* Leaf conductance to sensible heat, per unit all-sided LAI */}
00134     gl\_sh = gl\_bl;
00135     
00136     \textcolor{comment}{/* Canopy conductance to evaporated water vapor */}
00137     gc\_e\_wv = gl\_e\_wv * proj\_lai;
00138     
00139     \textcolor{comment}{/* Canopy conductane to sensible heat */}
00140     gc\_sh = gl\_sh * proj\_lai;
00141     
00142     cwe = trans = 0.0;
00143     
00144     \textcolor{comment}{/* Assign values in pmet\_in that don't change */}
00145     pmet\_in.ta = tday;
00146     pmet\_in.pa = metv->pa;
00147     pmet\_in.vpd = vpd;
00148     
00149     \textcolor{comment}{/* Canopy evaporation, if any water was intercepted */}
00150     \textcolor{comment}{/* Calculate Penman-Monteith evaporation, given the canopy conductances to}
00151 \textcolor{comment}{    evaporated water and sensible heat.  Calculate the time required to }
00152 \textcolor{comment}{    evaporate all the canopy water at the daily average conditions, and }
00153 \textcolor{comment}{    subtract that time from the daylength to get the effective daylength for}
00154 \textcolor{comment}{    transpiration. */}
00155     \textcolor{keywordflow}{if} (canopy\_w)
00156     \{
00157         \textcolor{comment}{/* assign appropriate resistance and radiation for pmet\_in */}
00158         pmet\_in.rv = 1.0/gc\_e\_wv;
00159         pmet\_in.rh = 1.0/gc\_sh;
00160         pmet\_in.irad = metv->swabs;
00161         
00162         \textcolor{comment}{/* call penman-monteith function, returns e in kg/m2/s */}
00163         \textcolor{keywordflow}{if} (\hyperlink{canopy__et_8c_a5c8d396ed94c7e0f3f4c80dd3ecd046c}{penmon}(&pmet\_in, 0, &e))
00164         \{
00165             \hyperlink{bgc__io_8c_af287cce6e2aede1ce337de9319e80d0d}{bgc\_printf}(BV\_ERROR, \textcolor{stringliteral}{"Error: penmon() for canopy evap... \(\backslash\)n"});
00166             ok=0;
00167         \}
00168         
00169         \textcolor{comment}{/* calculate the time required to evaporate all the canopy water */}
00170         e\_dayl = canopy\_w/e;
00171         
00172         \textcolor{keywordflow}{if} (e\_dayl > dayl)  
00173         \{
00174             \textcolor{comment}{/* day not long enough to evap. all int. water */}
00175             trans = 0.0;    \textcolor{comment}{/* no time left for transpiration */}
00176             cwe = e * dayl;   \textcolor{comment}{/* daylength limits canopy evaporation */}
00177         \}
00178         \textcolor{keywordflow}{else}                
00179         \{
00180             \textcolor{comment}{/* all intercepted water evaporated */}
00181             cwe = canopy\_w;
00182             
00183             \textcolor{comment}{/* adjust daylength for transpiration */}
00184             t\_dayl = dayl - e\_dayl;
00185              
00186             \textcolor{comment}{/* calculate transpiration using adjusted daylength */}
00187             \textcolor{comment}{/* first for sunlit canopy fraction */}
00188             pmet\_in.rv = 1.0/gl\_t\_wv\_sun;
00189             pmet\_in.rh = 1.0/gl\_sh;
00190             pmet\_in.irad = metv->swabs\_per\_plaisun;
00191             \textcolor{keywordflow}{if} (\hyperlink{canopy__et_8c_a5c8d396ed94c7e0f3f4c80dd3ecd046c}{penmon}(&pmet\_in, 0, &t))
00192             \{
00193                 \hyperlink{bgc__io_8c_af287cce6e2aede1ce337de9319e80d0d}{bgc\_printf}(BV\_ERROR, \textcolor{stringliteral}{"Error: penmon() for adjusted transpiration... \(\backslash\)n"});
00194                 ok=0;
00195             \}
00196             trans\_sun = t * t\_dayl * epv->plaisun;
00197             
00198             \textcolor{comment}{/* next for shaded canopy fraction */}
00199             pmet\_in.rv = 1.0/gl\_t\_wv\_shade;
00200             pmet\_in.rh = 1.0/gl\_sh;
00201             pmet\_in.irad = metv->swabs\_per\_plaishade;
00202             \textcolor{keywordflow}{if} (\hyperlink{canopy__et_8c_a5c8d396ed94c7e0f3f4c80dd3ecd046c}{penmon}(&pmet\_in, 0, &t))
00203             \{
00204                 \hyperlink{bgc__io_8c_af287cce6e2aede1ce337de9319e80d0d}{bgc\_printf}(BV\_ERROR, \textcolor{stringliteral}{"Error: penmon() for adjusted transpiration... \(\backslash\)n"});
00205                 ok=0;
00206             \}
00207             trans\_shade = t * t\_dayl * epv->plaishade;
00208             trans = trans\_sun + trans\_shade;
00209         \}
00210     \}    \textcolor{comment}{/* end if canopy\_water */}
00211     
00212     \textcolor{keywordflow}{else} \textcolor{comment}{/* no canopy water, transpiration with unadjusted daylength */}
00213     \{
00214         \textcolor{comment}{/* first for sunlit canopy fraction */}
00215         pmet\_in.rv = 1.0/gl\_t\_wv\_sun;
00216         pmet\_in.rh = 1.0/gl\_sh;
00217         pmet\_in.irad = metv->swabs\_per\_plaisun;
00218         \textcolor{keywordflow}{if} (\hyperlink{canopy__et_8c_a5c8d396ed94c7e0f3f4c80dd3ecd046c}{penmon}(&pmet\_in, 0, &t))
00219         \{
00220             \hyperlink{bgc__io_8c_af287cce6e2aede1ce337de9319e80d0d}{bgc\_printf}(BV\_ERROR, \textcolor{stringliteral}{"Error: penmon() for adjusted transpiration... \(\backslash\)n"});
00221             ok=0;
00222         \}
00223         trans\_sun = t * dayl * epv->plaisun;
00224         
00225         \textcolor{comment}{/* next for shaded canopy fraction */}
00226         pmet\_in.rv = 1.0/gl\_t\_wv\_shade;
00227         pmet\_in.rh = 1.0/gl\_sh;
00228         pmet\_in.irad = metv->swabs\_per\_plaishade;
00229         \textcolor{keywordflow}{if} (\hyperlink{canopy__et_8c_a5c8d396ed94c7e0f3f4c80dd3ecd046c}{penmon}(&pmet\_in, 0, &t))
00230         \{
00231             \hyperlink{bgc__io_8c_af287cce6e2aede1ce337de9319e80d0d}{bgc\_printf}(BV\_ERROR, \textcolor{stringliteral}{"Error: penmon() for adjusted transpiration... \(\backslash\)n"});
00232             ok=0;
00233         \}
00234         trans\_shade = t * dayl * epv->plaishade;
00235         trans = trans\_sun + trans\_shade;
00236     \}
00237         
00238     \textcolor{comment}{/* assign water fluxes, all excess not evaporated goes}
00239 \textcolor{comment}{    to soil water compartment */}
00240     wf->canopyw\_evap = cwe;
00241     wf->canopyw\_to\_soilw = canopy\_w - cwe;
00242     wf->soilw\_trans = trans;
00243     
00244     \textcolor{comment}{/* assign leaf-level conductance to transpired water vapor, }
00245 \textcolor{comment}{    for use in calculating co2 conductance for farq\_psn() */}
00246     epv->gl\_t\_wv\_sun = gl\_t\_wv\_sun; 
00247     epv->gl\_t\_wv\_shade = gl\_t\_wv\_shade;
00248     
00249     \textcolor{comment}{/* assign verbose output variables if requested */}
00250     \textcolor{keywordflow}{if} (verbose)
00251     \{
00252         epv->m\_ppfd\_sun  = m\_ppfd\_sun;
00253         epv->m\_ppfd\_shade  = m\_ppfd\_shade;
00254         epv->m\_psi   = m\_psi;
00255         epv->m\_co2   = m\_co2;
00256         epv->m\_tmin  = m\_tmin;
00257         epv->m\_vpd   = m\_vpd;
00258         epv->m\_final\_sun = m\_final\_sun;
00259         epv->m\_final\_shade = m\_final\_shade;
00260         epv->gl\_bl   = gl\_bl;
00261         epv->gl\_c    = gl\_c;
00262         epv->gl\_s\_sun   = gl\_s\_sun;
00263         epv->gl\_s\_shade = gl\_s\_shade;
00264         epv->gl\_e\_wv = gl\_e\_wv;
00265         epv->gl\_sh   = gl\_sh;
00266         epv->gc\_e\_wv = gc\_e\_wv;
00267         epv->gc\_sh   = gc\_sh;
00268     \}
00269     
00270     \textcolor{keywordflow}{return} (!ok);
00271 \}
00272 
\hypertarget{canopy__et_8c_source_l00273}{}\hyperlink{canopy__et_8c_a5c8d396ed94c7e0f3f4c80dd3ecd046c}{00273} \textcolor{keywordtype}{int} \hyperlink{canopy__et_8c_a5c8d396ed94c7e0f3f4c80dd3ecd046c}{penmon}(\textcolor{keyword}{const} pmet\_struct* in, \textcolor{keywordtype}{int} out\_flag,   \textcolor{keywordtype}{double}* et)
00274 \{
00275     \textcolor{comment}{/*}
00276 \textcolor{comment}{    Combination equation for determining evaporation and transpiration. }
00277 \textcolor{comment}{    }
00278 \textcolor{comment}{    For output in units of (kg/m2/s)  out\_flag = 0}
00279 \textcolor{comment}{    For output in units of (W/m2)     out\_flag = 1}
00280 \textcolor{comment}{   }
00281 \textcolor{comment}{    INPUTS:}
00282 \textcolor{comment}{    in->ta     (deg C)   air temperature}
00283 \textcolor{comment}{    in->pa     (Pa)      air pressure}
00284 \textcolor{comment}{    in->vpd    (Pa)      vapor pressure deficit}
00285 \textcolor{comment}{    in->irad   (W/m2)    incident radient flux density}
00286 \textcolor{comment}{    in->rv     (s/m)     resistance to water vapor flux}
00287 \textcolor{comment}{    in->rh     (s/m)     resistance to sensible heat flux}
00288 \textcolor{comment}{}
00289 \textcolor{comment}{    INTERNAL VARIABLES:}
00290 \textcolor{comment}{    rho    (kg/m3)       density of air}
00291 \textcolor{comment}{    CP     (J/kg/degC)   specific heat of air}
00292 \textcolor{comment}{    lhvap  (J/kg)        latent heat of vaporization of water}
00293 \textcolor{comment}{    s      (Pa/degC)     slope of saturation vapor pressure vs T curve}
00294 \textcolor{comment}{}
00295 \textcolor{comment}{    OUTPUT:}
00296 \textcolor{comment}{    et     (kg/m2/s)     water vapor mass flux density  (flag=0)}
00297 \textcolor{comment}{    et     (W/m2)        latent heat flux density       (flag=1)}
00298 \textcolor{comment}{    */}
00299 
00300     \textcolor{keywordtype}{int} ok=1;
00301     \textcolor{keywordtype}{double} ta;
00302     \textcolor{keywordtype}{double} rho,lhvap,s;
00303     \textcolor{keywordtype}{double} t1,t2,pvs1,pvs2,e,tk;
00304     \textcolor{keywordtype}{double} rr,rh,rhr,rv;
00305     \textcolor{keywordtype}{double} dt = 0.2;     \textcolor{comment}{/* set the temperature offset for slope calculation */}
00306    
00307     \textcolor{comment}{/* assign ta (Celsius) and tk (Kelvins) */}
00308     ta = in->ta;
00309     tk = ta + 273.15;
00310         
00311     \textcolor{comment}{/* calculate density of air (rho) as a function of air temperature */}
00312     rho = 1.292 - (0.00428 * ta);
00313     
00314     \textcolor{comment}{/* calculate resistance to radiative heat transfer through air, rr */}
00315     rr = rho * CP / (4.0 * SBC * (tk*tk*tk));
00316     
00317     \textcolor{comment}{/* resistance to convective heat transfer */}
00318     rh = in->rh;
00319     
00320     \textcolor{comment}{/* resistance to latent heat transfer */}
00321     rv = in->rv;
00322     
00323     \textcolor{comment}{/* calculate combined resistance to convective and radiative heat transfer,}
00324 \textcolor{comment}{    parallel resistances : rhr = (rh * rr) / (rh + rr) */}
00325     rhr = (rh * rr) / (rh + rr);
00326 
00327     \textcolor{comment}{/* calculate latent heat of vaporization as a function of ta */}
00328     lhvap = 2.5023e6 - 2430.54 * ta;
00329 
00330     \textcolor{comment}{/* calculate temperature offsets for slope estimate */}
00331     t1 = ta+dt;
00332     t2 = ta-dt;
00333     
00334     \textcolor{comment}{/* calculate saturation vapor pressures at t1 and t2 */}
00335     pvs1 = 610.7 * exp(17.38 * t1 / (239.0 + t1));
00336     pvs2 = 610.7 * exp(17.38 * t2 / (239.0 + t2));
00337 
00338     \textcolor{comment}{/* calculate slope of pvs vs. T curve, at ta */}
00339     s = (pvs1-pvs2) / (t1-t2);
00340     
00341     \textcolor{comment}{/* calculate evaporation, in W/m^2  */}
00342     e = ( ( s * in->irad ) + ( rho * CP * in->vpd / rhr ) ) /
00343         ( ( ( in->pa * CP * rv ) / ( lhvap * EPS * rhr ) ) + s );
00344     
00345     \textcolor{comment}{/* return either W/m^2 or kg/m^2/s, depending on out\_flag */}    
00346     \textcolor{keywordflow}{if} (out\_flag)
00347         *et = e;
00348     
00349     \textcolor{keywordflow}{if} (!out\_flag)
00350         *et = e / lhvap;
00351     
00352     \textcolor{keywordflow}{return} (!ok);
00353 \}
\end{DoxyCode}
